\textit{Objective.} Serving as a channel for communication with locked-in patients or control of prostheses, sensorimotor brain-computer interfaces (BCIs) decode imaginary movements from the recorded activity of the user's brain. However, many individuals remain unable to control the BCI, and the underlying mechanisms are unclear. The user's BCI performance was previously shown to correlate with the resting-state signal-to-noise ratio (SNR) of the mu rhythm and the phase synchronization (PS) of the mu rhythm between sensorimotor areas. Yet, these predictors of performance were primarily evaluated in a single BCI session, while the longitudinal aspect remains rather uninvestigated. In addition, different analysis pipelines were used to estimate PS in source space, potentially hindering the reproducibility of the results. \textit{Approach.} To systematically address these issues, we performed an extensive validation of the relationship between pre-stimulus SNR, PS, and session-wise BCI performance using a publicly available dataset of 62 human participants performing up to 11 sessions of BCI training. We performed the analysis in sensor space using the surface Laplacian and in source space by combining 24 processing pipelines in a multiverse analysis. This way, we could investigate how robust the observed effects were to the selection of the pipeline. \textit{Main results.} Our results show that SNR had both between- and within-subject effects on BCI performance for the majority of the pipelines. In contrast, the effect of PS on BCI performance was less robust to the selection of the pipeline and became non-significant after controlling for SNR. \textit{Significance.} Taken together, our results demonstrate that changes in neuronal connectivity within the sensorimotor system are not critical for learning to control a BCI, and interventions that increase the SNR of the mu rhythm might lead to improvements in the user's BCI performance.
