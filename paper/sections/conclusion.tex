\section{Conclusion}

Overall, we observed that SNR affected BCI performance both on the between- and within-subject levels: Participants with higher SNR tended to perform better, and the same participant also tended to perform better on the days when SNR was higher. Therefore, interventions that are suitable for increasing SNR might lead to an improvement in performance. Additionally, multiverse analyses allowed us to analyze the robustness of the investigated effects to the selection of the pipeline for source space analysis. The results suggest that SNR was a primary factor of the observed performance variability as it robustly predicted accuracy and covaried with phase synchronization (PS). On the contrary, the effects of PS became non-significant after controlling for SNR and were less consistent across different pipelines. We observed no evidence of longitudinal changes in SNR and only weak evidence of an increase in the across-hemisphere coherence during the training. At the same time, SNR and PS values were significantly affected by the selection of the pipeline for source space analysis. Therefore, it is necessary to include several pipelines in the analysis to assess how robust the observed effects are and how high the between-pipeline variability is. This paper can serve as a template for future multiverse analyses as it represents an end-to-end fully repeatable pipeline from raw data to the publishable report, and all the underlying data and scripts are publicly available.