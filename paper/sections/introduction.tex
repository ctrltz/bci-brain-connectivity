\section{Introduction}

A brain-computer interface (BCI) is a system that decodes the intentions of the user based on the recorded activity of their brain and provides commands to external devices (e.g., prostheses; \cite{Wolpaw2002}). These systems have many potential applications ranging from the clinical ones, such as providing a communication pathway for locked-in patients \citep{Chaudhary2016} or ameliorating symptoms in patients with stroke and Parkinson's disease \citep{McFarland2017}, to the research ones, such as the detection of mental states and the facilitation of actions in healthy humans \citep{Blankertz2010review, Blankertz2016}. Often, BCIs are based on magnetoencephalographic (MEG) or electroencephalographic (EEG) recordings of brain activity. MEG and EEG (M/EEG) have high temporal resolution and provide multiple features of the ongoing or evoked brain activity that can be used as a control signal \citep{Abiri2019}. For example, BCI paradigms based on the P300 component of the evoked response or steady-state visual evoked responses (SSVEP) provide high information transfer rates for efficient communication \citep{Abiri2019}. However, these paradigms always require external stimuli to be presented, which makes the approach less flexible. In contrast, sensorimotor BCIs decode the imaginary movements of limbs or tongue that can be self-initiated and thus provide more flexibility \citep{Leeb2007, YuanHe2014, Scherer2018}. Decoding of the imaginary movements is often based on the modulation of power in the alpha (8 -- 13 Hz) and beta (13 -- 30 Hz) frequency ranges in sensorimotor brain areas, also referred to as event-related desynchronization or synchronization (ERD/ERS; \cite{PfurtschellerLopesDaSilva1999, Pfurtscheller1996}). Sensorimotor BCIs are also used to facilitate the recovery of motor functions during rehabilitation after a stroke \citep{Cervera2018, Kruse2020, Peng2022}.

\medskip

While BCI seems to be a promising approach with multiple clinical applications, some participants remain unable to control it \citep{AllisonNeuper2010}. Typically, participants complete several training sessions to learn to control a BCI. However, their performance in the task varies considerably, and on average around 20\% of the participants fail to learn the task \citep{Sannelli2019}. The mechanisms underlying successful modulation of brain activity for controlling a BCI are not clear yet. However, previous studies have identified several psychological \citep{Hammer2012, Jeunet2015} and neurophysiological \citep{Blankertz2010, Sugata2014, Samek2016, Vidaurre2020, Jorajuria2023} predictors of successful control of a sensorimotor BCI. These predictors allow pre-screening of participants to provide the full training only if the participant is likely to successfully control the BCI \citep{Sannelli2019}. 

\medskip

Neurophysiological predictors of successful BCI control also provide information about the features of brain activity (e.g., neuronal networks) that play a role in the success of BCI training. For example, the signal-to-noise ratio (SNR) of the sensorimotor mu rhythm during resting state was positively correlated ($r = 0.53$) with the online accuracy of sensorimotor BCI control \citep{Blankertz2010}. The SNR was defined as the maximal ratio of the periodic and aperiodic (1/f noise) components of the power spectrum in the 2-35 Hz frequency range. This predictor was later validated in an independent dataset with a similar experimental paradigm \citep{Acqualagna2016}. Moreover, several other neural correlates of performance in a sensorimotor BCI task are related to the SNR of the mu rhythm, for example, the performance potential factor \citep{Ahn2013} or the spectral entropy at C3 electrode during resting-state \citep{Zhang2015}. 

\medskip

Although SNR seems to be a well-established predictor of BCI performance, it is often investigated in the context of a single BCI session. However, the relationship between SNR and performance could change if participants with low SNR eventually learned the task or if the SNR changed throughout a multi-session BCI training. Therefore, it is crucial to validate this predictor in a longitudinal analysis, which is one of the aims of the current study.

\medskip

Other predictors of sensorimotor BCI performance include long-range temporal correlations \citep{Samek2016}, functional connectivity between sensorimotor brain regions \citep{Sugata2014, Vidaurre2020}, and the strength of mu vs. beta phase-phase coupling \citep{Jorajuria2023}. Connectivity-based predictors might be especially relevant since motor imagery involves activation of multiple interacting brain areas \citep{Solodkin2004, Halder2011, Hardwick2018}. The strength and the phase lag of these interactions can be quantified using various connectivity measures and then related to the performance in the sensorimotor BCI task. Thereby, connectivity could provide additional information about the underlying neuronal networks that is not reflected in the SNR.

\medskip

When considering M/EEG-based functional connectivity within the same (e.g., alpha/mu) frequency band, phase synchronization (PS) and amplitude envelope correlation (AEC) can reflect different properties of the underlying neuronal networks. Studies combining EEG and fMRI (functional Magnetic Resonance Imaging) have previously shown that the power of alpha and beta oscillations at C3 and C4 is negatively correlated with the blood-oxygen-level-dependent (BOLD) fMRI signal in sensorimotor areas during the execution of real and imaginary hand movements \citep{Ritter2009, Yuan2010}. Therefore, AEC primarily captures the low-frequency (below 0.1 Hz) dynamics of brain activity similar to the fMRI connectivity based on the BOLD signal \citep{Engel2013}. In contrast, phase synchronization between high-frequency (above 5 Hz) oscillations might reveal additional information that is only accessible with the high temporal resolution of M/EEG \citep{Engel2013}. In particular, phase synchronization was proposed to be a mechanism of efficient communication between neuronal populations \citep{Engel2001, Fries2005, PalvaPalva2007} and can reflect short-term changes in the functional organization of neuronal networks due to plasticity \citep{Engel2013}. Therefore, in the current study, we also investigated the role of phase synchronization of the sensorimotor mu (9-15 Hz) oscillations in the successful control of a sensorimotor BCI.

\medskip

Several studies have already applied various M/EEG-based phase synchronization measures in the context of sensorimotor BCI training. First, BCI performance was positively correlated with the imaginary part of coherency (ImCoh; \cite{Nolte2004}) of the mu rhythm between sensorimotor areas both before and during the trial \citep{Sugata2014, Vidaurre2020}. In addition, the phase locking value (PLV; \cite{Lachaux1999}) of alpha-band oscillations within the motor areas of the right hemisphere was higher for the successful participants in comparison to the unsuccessful ones \citep{Leeuwis2021}. Finally, in a whole-head analysis, \cite{Corsi2020} observed a global decrease in ImCoh during motor imagery compared to resting state. While phase synchronization seems to play a role in sensorimotor BCI training, the results were obtained using various PS measures and partially in the context of single-session experiments. To address these issues, we examined several PS measures and ran a longitudinal analysis of changes in phase synchronization and its relationship with the BCI performance.

\medskip

Studies investigating longitudinal changes in phase synchronization are scarce in the sensorimotor BCI literature. On the one hand,~\cite{Corsi2020} observed a progressive decrease of ImCoh during motor imagery in alpha and beta bands along sessions. On the other hand, the positive correlation between ImCoh and BCI performance in one session \citep{Sugata2014, Vidaurre2020} may suggest the entrainment of task-relevant networks throughout the training. However, in both cases, ImCoh reflects a mixture of the strength and the phase lag of the interaction between brain areas, which can only be disentangled with other PS measures, such as coherence. Therefore, further validation of these results in the longitudinal setting with multiple PS measures is necessary.

\medskip

In practice, the estimation of phase synchronization in M/EEG critically depends on the proper control for confounding factors \citep{Bastos2016}. In the current study, we focused on the effects of volume conduction and signal-to-noise ratio (SNR). To overcome these challenges, we used PS measures, which are insensitive to zero-lag interactions (e.g., ImCoh), and applied a correction for SNR in the statistical analysis.

\medskip

Furthermore, to obtain a higher spatial specificity of the estimated PS values, we performed the source space analysis using time courses of brain activity in particular regions of interest (ROIs). For this purpose, two-step processing pipelines are typically used \citep{SchoffelenGross2009}. First, inverse modeling is applied to reconstruct time courses of activity for individual sources within the cortex. Second, time courses of activity for all sources within the ROI are aggregated to extract one or several time courses of activity in the ROI. While multiple approaches exist for inverse modeling and extraction of ROI time series, there is no consensus on the most appropriate pipeline in the community. Previous studies have shown that the choice of methods for inverse modeling and extraction of ROI time series affects the estimated PS values in real and simulated data \citep{Mahjoory2017, Pellegrini2023}. Therefore, multiple pipelines should be considered simultaneously to arrive at a valid conclusion about genuine neuronal connectivity based on M/EEG data.

\medskip

To address the multitude of possible pipelines while analyzing SNR and phase synchronization as predictors of BCI performance, we ran a multiverse analysis \citep{Steegen2016} using several pipelines for extraction of ROI time series. While results are typically reported only for one or a few of many possible pipelines, the idea of the multiverse analysis is to consider a set of reasonable pipelines and report the results for all of the considered options. This way, one can not only analyze the variability of the estimated PS values similar to \cite{Mahjoory2017} but also assess the robustness of the observed effects (e.g., on BCI performance) to the selection of the pipeline. More pronounced effects should be more robust to changes in the processing pipeline, and including several pipelines in the analysis may reveal important information about the influence of different processing steps on the observed results.

\medskip

Overall, in the current study, we aimed to validate and extend the findings about the effects of SNR and PS of the mu rhythm on BCI performance in a publicly available longitudinal dataset \citep{Stieger2021_dataset}. We focused on four sensorimotor ROIs corresponding to the primary motor and somatosensory cortices. These ROIs were previously shown to be the most involved in the BCI training based on imaginary movements \citep{Samek2016, Vidaurre2020, Nierhaus2021}. In the current analysis, we aimed to address the following research questions:

\begin{enumerate}
    \item Do SNR and PS predict performance not just in one but also in multiple training sessions?
    \item Do SNR and PS change over time due to BCI training?
    \item Are SNR, PS, and the observed effects for questions 1 and 2 robust to the selection of processing steps in the source space analysis?
\end{enumerate}

To touch upon the open questions regarding the multitude of existing approaches for source space analysis and estimation of phase synchronization, we considered a set of existing methods and performed a multiverse analysis to capture the between-pipeline variability in estimated values of SNR and PS, their effects on BCI performance, and longitudinal changes over time. In addition, to ensure the end-to-end repeatability of the results, we designed the analysis pipeline to automatically include the results in a publishable report, which, as we hope, will be useful as a template for future studies involving a multitude of different analysis pipelines.