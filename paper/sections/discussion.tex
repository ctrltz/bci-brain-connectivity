\section{Discussion} \label{sec:discuss}

In the current study, we investigated the role of SNR and phase synchronization (PS) of the mu rhythm in sensorimotor brain areas as predictors of BCI performance in a multi-session training using a publicly available dataset \citep{Stieger2021_dataset}. The dataset contained EEG recordings from a multi-session BCI training based on a cursor control paradigm. The performance of the participants was assessed with the accuracy of completed trials and improved significantly for some but not all participants throughout the training. Overall, the mean accuracy was comparable to other BCI studies and similar to the 70\% threshold, which is commonly used to identify good performers (e.g., in \cite{Sannelli2019} and \cite{Leeuwis2021}). While the increase in group-average performance was not prominent between sessions 2 and 10, a considerable level of intra-individual variability of performance was observed. We used linear mixed models to account for this variability and investigate the relationship between SNR, PS, and BCI performance, as well as longitudinal changes in SNR and PS due to training. We performed the analysis in sensor space using the surface Laplacian and in source space by combining several processing pipelines in a multiverse analysis. In the following, we discuss the results of the study and their prospective applications.

\subsection{SNR in the context of sensorimotor BCI training}

Previous studies have shown that the signal-to-noise ratio of the mu rhythm estimated at the Laplace-filtered channels C3 and C4 correlated with the accuracy of BCI control \citep{Blankertz2010, Acqualagna2016, Sannelli2019}. We also observed a positive correlation between Laplace SNR and accuracy after averaging over all sessions, which reflects a between-subject effect of SNR on performance. Additionally, we observed a within-subject effect of Laplace SNR on accuracy. That is, not only do participants with a higher SNR of the mu rhythm tend to perform better, but the same participant tends to perform better on the days when SNR is higher as well. 

\medskip

In general, larger SNR is associated with stronger lateralization of the mu rhythm during imaginary movements, leading to a higher classification or control accuracy \citep{Maeder2012}. Our results show that this finding, previously observed primarily for the single experimental sessions, generalizes to longitudinal settings and has two important consequences. First, changes in overall performance should be controlled for the changes in SNR to make conclusions about other possible neurophysiological factors. Second, experimental adjustments leading to an increase in SNR might also translate to a performance improvement. 

\medskip

It is important to note that SNR might affect BCI performance at least in two different ways. On the one hand, participants with low SNR of the mu rhythm might not be able to perform vivid imaginary movements and modulate their brain activity strongly enough. In this case, ``quasi-movements'' (i.e., real movements minimized to such an extent that they cannot be detected by objective measures) could be used to train participants to perform the motor imagery better \citep{Nikulin2008}. On the other hand, high SNR of the mu rhythm might translate into a more reliable feedback signal, which would in turn allow participants to train the imaginary movements more efficiently. If this is the case, training for participants with low SNR of the mu rhythm could be based on other features of brain activity that might provide a higher SNR. For example, \cite{Tao2021} have shown that motor imagery led to a decrease in inter-trial phase coherence during steady-state electrical stimulation of the median nerve. 

\medskip

Moreover, there is still a considerable amount of unexplained variance in BCI performance, which could be attributed to other psychological (such as motivation or concentration) and lifestyle (sports or musical instrument training) factors. These factors remain a subject of extensive research in the BCI community \citep{Hammer2012, Jeunet2015} and could also be manipulated to improve BCI performance.

\medskip

Furthermore, we investigated whether Laplace SNR itself could change throughout sensorimotor BCI training but observed no evidence of longitudinal changes. This result could be related to the structure of the cursor control tasks. Typically, post-effects of BCI or neurofeedback are observed when the whole training is based on a fixed direction of modulation of brain activity, for example, up-regulation of alpha power \citep{Zoefel2011}. In contrast, cursor control tasks in the analyzed dataset always contained trials with opposite directions of modulation of the mu rhythm (left- vs. right-hand imaginary movements or motor imagery vs. relaxation). Therefore, on average, task-related modulation of the mu rhythm may not have a cumulative effect across many sessions. In addition, \cite{Popov2023} have also reported an excellent (ICC = 0.83) test-retest reliability of the periodic component of the alpha power in the sensorimotor regions. While this finding goes in line with the absence of longitudinal changes in SNR in the analyzed dataset, there still was a within-subject effect of SNR on BCI performance. This result could be explained if SNR is a trait feature that is affected by measurement-related effects (e.g., different placement of the electrodes) on different training days. Nevertheless, measurement-related effects could, in turn, make the detection of longitudinal changes in SNR harder.

\medskip

The absence of longitudinal changes in SNR is critical for discussing changes in other measures that were shown to be correlated with SNR such as phase synchronization or long-range temporal correlations \citep{Samek2016, Vidaurre2020}. Since a decrease in SNR typically leads to the attenuation of the aforementioned measures, their changes (e.g., due to learning, arousal, etc.) should be controlled for the concurrent changes in SNR.

\medskip

In our study, both the positive effect of SNR on accuracy and the absence of longitudinal changes in SNR were robust to the selection of the processing steps in the multiverse analysis, as the results were the same for all of the considered pipelines. Taken together with all the existing evidence for the role of SNR in BCI training, this result might suggest that the effect of SNR on accuracy is strong enough to overcome the variability in the estimation of SNR across different pipelines.

\subsection{Phase synchronization in the context of sensorimotor BCI training}

In the current study, we analyzed three linear PS measures to combine the interpretability of coherence (as it reflects the strength of interaction) and robustness to zero-lag interactions provided by ImCoh and LagCoh. The estimation of PS was performed in the source space, and several processing pipelines were combined in a multiverse analysis to assess the variability of the PS values and associated statistical effects. For most pipelines, we observed a peak in the mu range of the PS spectra, which reflects an interaction that is specific to mu oscillations.

\medskip

In line with several previous studies \citep{Bayraktaroglu2013, Vidaurre2020}, we observed a positive correlation between the values of SNR and phase synchronization. On the one hand, higher SNR improves phase estimation and may spuriously lead to higher values of PS \citep{MuthukumaraswamySingh2011}. On the other hand, a higher PS between two neuronal populations is likely to co-occur with a higher level of synchronization within the populations, which would be manifested in higher SNR values \citep{Schneider2021}. Most likely, both factors contribute to a positive correlation between SNR and PS values. This correlation was very robust to the selection of the pipeline for ImCoh and LagCoh, which are not sensitive to zero-lag spurious interactions due to volume conduction. Effects of SNR on coherence were less consistent, which could be related to the remaining spatial leakage (i.e., signal mixing), especially in the case of nearby regions within the same hemisphere. Overall, our findings confirm that it is necessary to account for changes in SNR when analyzing phase synchronization.

\medskip

We observed a significant positive within-subject effect of within- and across-hemisphere ImCoh and LagCoh on BCI performance. It was significant in the joint analysis and for a few separate pipelines in the split analysis. While this finding goes in line with the results of \citep{Vidaurre2020}, we observed no evidence for a between-subject effect (Fig. \ref{supp-fig:multiverse_connectivity_performance_between}), which could serve as a direct replication. After correction for SNR, the effect of ImCoh but not LagCoh on BCI performance remained significant in the joint multiverse analysis. At the same time, the same effect could not be detected by any of the considered pipelines in the split analysis. Therefore, our results suggest that phase synchronization was not related to BCI performance in the analyzed dataset. While motor imagery leads to a modulation of amplitude (ERD/ERS), it might not necessarily require phase synchronization as strongly as other tasks involving precise bilateral coordination \citep{Shih2021}. However, this result may also be caused by differences in the calculation of real-time feedback. In the dataset analyzed by \cite{Vidaurre2020}, feedback is calculated using a CSP-based spatial filter, which is optimal for extracting task-related differences in power. In the analyzed dataset, surface Laplacian is applied to channels C3 and C4 for spatial filtering, and the resulting signal may still contain contributions from other sources, e.g., occipital alpha rhythm \citep{Blankertz2010}.

\medskip

Despite not showing high consistency between pipelines (Fig. \ref{supp-fig:multiverse_connectivity_longitude}), an increase in across-hemisphere coherence throughout the training was significant in the joint multiverse analysis. This result could speak in favor of the optimization of the interaction between motor areas due to the training. However, since the same effect was only observed in two pipelines for LagCoh and not at all for ImCoh, there is not enough evidence or statistical power to conclude that this increase is driven by a genuine neural interaction. 

\medskip

Overall, the findings related to phase synchronization were not as robust to the selection of the pipeline as they were for SNR. Hence, along with the recommendation from \cite{Mahjoory2017}, it is necessary to include at least several analysis pipelines to account for the between-pipeline variability of PS values.

\subsection{Effects of the processing methods on the estimated values of SNR and PS}

The multiverse analysis also allowed us to compare SNR and PS values that were obtained by applying different combinations of methods for source space analysis to the same data. Since there is no ground truth available for real data, this comparison does not allow us to determine which methods work better or worse \citep{FeuerriegelBode2022}. Nevertheless, below we describe several observations that could be validated in simulations and used in future studies.

\paragraph{Inverse Modeling} 

As shown in Fig. \ref{supp-fig:snr_band_inverse}, for broadband pipelines, SNR was higher on average when LCMV was used for inverse modeling compared to eLORETA. Since LCMV is a data-driven approach, it might better adapt to different subjects and sessions and thereby extract oscillatory activity with higher SNR than eLORETA. Surprisingly, the difference in SNR between pipelines with LCMV and eLORETA was especially prominent for low values of SNR. However, it is not clear whether the improvement in the SNR of the extracted signal is due to better extraction of activity from the investigated ROIs or the remaining spatial leakage from other ROIs. After correction for SNR, LCMV led to a decrease in LagCoh and an increase in across-hemisphere coherence compared to eLORETA (Fig. \ref{fig:snr_connectivity}A). Previous studies \citep{Mahjoory2017, Pellegrini2023} also observed the impact of the inverse method on the estimated PS values. While the reasons behind this effect are not clear, it is important to note that the selection of the inverse method also played a role in the split multiverse analysis. In particular, the effects of within-hemisphere ImCoh and LagCoh on BCI performance were significant only for pipelines that included LCMV (Fig. \ref{fig:multiverse_connectivity_performance_within}A).

\paragraph{Extraction of ROI Time Series}

ROI time series were obtained by aggregation of time series of individual sources within the ROI, and the selection of the aggregation method affected most PS measures. In particular, for the within-hemisphere case, the first SVD component seemed to capture the remaining effects of volume conduction to a great extent, as indicated by the lack of a peak in the spectra of coherence (Fig. \ref{fig:snr_connectivity}A) and the values of coherence that are very close to 1 (Fig. \ref{fig:pipeline_effects_highlights}C). In contrast, when three SVD components were used for the calculation of the PS, a peak in the spectra was present, and coherence was generally lower, while ImCoh had higher values. This result might be caused by the averaging of pairwise connectivity values between different SVD components, which is more likely to result in a non-zero phase lag. Still, by including more than one component per ROI in the analysis, one might ensure that a genuine interaction between ROIs is captured. This observation goes in line with the recommendation to consider 3-4 SVD components per ROI from \citep{Pellegrini2023}. Averaging with sign flip also seemed to capture the remaining effects of volume conduction less, as reflected by lower coherence and higher ImCoh.

\paragraph{Filtering}

Fig. \ref{supp-fig:snr_inverse_band} shows that for pipelines that included eLORETA as the method for inverse modeling, SNR was higher when the data were filtered in a narrow band. Since eLORETA is a data-independent method, it is not affected by the frequency range of the data. Instead, filtering matters for SVD being the only data-dependent method in the subsequent steps of the pipelines. This result supports the observation by \cite{Chalas2022} that SVD prioritizes low frequencies that explain more variance in the M/EEG signals. At the same time, the effects of filtering on all PS measures were not significant before and after controlling for SNR.

\paragraph{ROI Definition}

We investigated whether reducing anatomical definitions of ROIs to a subset of task-relevant sources could make the estimated SNR and PS values even more task-specific. The definition of the ROI played a different role in the estimation of within-hemisphere and across-hemisphere PS. In the within-hemisphere case, the task-based definition reduced the size of the ROIs and variability in the reconstructed time series of individual sources. Thereby, the effects of volume conduction became pronounced even stronger (higher coherence and lower ImCoh). In the across-hemisphere case, the distance between task-based ROIs was higher than between the anatomical ones, and the observed decrease in coherence could reflect less pronounced volume conduction.

\paragraph{Interactions}

Although we included all possible two-way interactions between processing steps in the model, only two of them were significant after correction for multiple comparisons (Tab. \ref{supp-tab:pipeline_effects_interactions}). First, there was a significant interaction between filtering and inverse modeling steps, which implies that there was a difference between PS and SNR values depending on whether eLORETA and LCMV were fit to broadband or narrowband data. In the analyzed dataset, narrowband LCMV led to lower SNR for all combinations of subsequent processing steps (Fig. \ref{supp-fig:snr_inverse_band}). Second, for within-hemisphere PS values, there was a significant interaction between ROI definition and method for extraction of ROI time series. When three SVD components are used to describe activity in each ROI, the definition of the ROI (anatomical or task-based) seems to play a smaller role than for averaging with sign flip or a single SVD component (Fig. \ref{supp-fig:coh_within_mask_roi_method}).

\medskip

Overall, all processing steps affected SNR or some of the PS measures, highlighting the importance of considering several pipelines for source space analysis. Both data-dependent methods (LCMV and SVD) led to different results when fit to broadband and narrowband data. The definition of the ROI and the method for extraction of ROI time series seemed to make the biggest difference in terms of remaining volume conduction but also affected PS measures that are insensitive to zero-lag interactions.

\subsection{Limitations}
    
The current analysis was limited to four sensorimotor ROIs and did not include the whole-brain connectivity patterns, for example, as performed in \citep{Corsi2020}. This selection was based on previous studies showing that motor imagery BCI primarily leads to activation of the sensorimotor areas that we analyzed \citep{Nierhaus2021}. These ROIs contained the highest amount of task-relevant sources in the analyzed dataset as well (Tab. \ref{supp-tab:csp_sources_per_roi}), thereby additionally validating the selection. As described before, there are several open questions regarding the estimation of PS, correlation between PS and behavior, correction for the effect of SNR and PS, and interpretation of the results. Analyzing only selected ROIs made it feasible to address these challenges by considering several options for each question.

\medskip

There also exist other methods that were not included in the multiverse analysis to ensure computational feasibility, e.g., dynamic imaging of coherent sources (DICS; \cite{DICS_Gross2001}) for inverse modeling or fidelity weighting \citep{Korhonen2014} for aggregation of ROI time series. However, the amount of pipelines considered in the current analysis already provides additional insights compared to a single pipeline. Still, it is important to keep in mind that even if similar results are obtained with multiple pipelines, it does not directly imply the genuineness of these results.

\medskip 

The final limitation is related to the longitudinal analysis. While the group-level improvement in performance was significant, group-average accuracy was similar across most sessions, which might reflect little evidence of training effects. Nevertheless, we utilized the observed within-subject variability and employed linear mixed models to estimate the effects of interest.

\subsection{Future Directions}

In the current study, we focused on predicting the BCI performance from the dynamics of mu rhythm in sensorimotor areas during the inter-trial interval, thereby testing the existing resting-state findings \citep{Blankertz2010, Vidaurre2020} in a longitudinal setting. Future studies could consider using task data and investigating other brain regions --- for example, connectivity between visual and sensorimotor brain areas could reflect the processing of the visual feedback during motor imagery. Moreover, combining datasets with several BCI paradigms \citep{MOABB2024} could provide additional insights into feedback processing that generalize beyond a single paradigm (in this case, motor imagery).

\medskip

Multiverse analysis seems to be a promising approach for incorporating the variability between different processing pipelines into the analysis. In the context of M/EEG connectivity, it could be interesting to evaluate the robustness of amplitude coupling measures \citep{Hipp2012, Brookes2012} and network measures of brain connectivity \citep{RubinovSporns2010} to the selection of the pipeline for source space analysis.
